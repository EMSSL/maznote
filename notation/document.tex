\documentclass{article}
\usepackage[utf8]{inputenc}
\usepackage{MazzoleniNotation_rev3}
\usepackage{tabularx}

\usepackage[margin=1in]{geometry}
% \usepackage[showframe]{geometry}


\title{EMSSL-syntax, rev2}
\author{Christopher D. yoder}
\date{September 29$^{th}$ 2020}

\begin{document}
\maketitle

\noindent
Angular momentum: \verb|\AngMom{derivative-frame}{reference-point}{point}|\\
Example: \verb|\AngMom{O}{P}{A}|\\
Output:
\begin{equation*}
    \AngMom{O}{P}{A}
\end{equation*}

\noindent
Acceleration: \verb|\AcclVec{derivative-frame}{point}{with-respect-to}|\\
Example: \verb|\AcclVec{O}{A}{Q}|\\
Output:
\begin{equation*}
    \AcclVec{O}{A}{Q}
\end{equation*}

\noindent
Angular Acceleration: \verb|\AngAccl{frame-1}{frame-2}|\\
Example: \verb|\AngAccl{O}{B}|\\
Output:
\begin{equation*}
    \AngAccl{O}{B}
\end{equation*}

\noindent
Angular Velocity: \verb|\AngAccl{frame-1}{frame-2}|\\
Example: \verb|\AngAccl{O}{B}|\\
Output:
\begin{equation*}
    \AngAccl{O}{B}
\end{equation*}

\noindent
Cross two vectors: \verb|\Cross{vector-1}{vector-2}|\\
Example: \verb|\Cross{A}{B}|\\
Output:
\begin{equation*}
    \Cross{A}{B}
\end{equation*}

\noindent
Cross product expression: \verb|\CrossProd{frame}{a}{b}{c}{d}{e}{f}|\\
Example: \verb|\CrossProd{frame}{a}{b}{c}{d}{e}{f}|\\
Output:
\begin{equation*}
    \CrossProd{O}{a}{b}{c}{d}{e}{f}
\end{equation*}

\noindent
Derivative: \verb|\Deriv[order]{with-respect-to-variable}|\\
Example: \verb|\Deriv[]{x}|\\
Output:
\begin{equation*}
    \Deriv[]{x}
\end{equation*}
Example: \verb|\Deriv[3]{t}|\\
\begin{equation*}
    \Deriv[3]{t}
\end{equation*}

\noindent
Direction cosine matrix: \verb|\DirectCosMat|\\
Example: \verb|\DirectCosMat|\\
Output:
\begin{equation*}
    \DirectCosMat
\end{equation*}

\noindent
Sum of external forces: \verb|\Fext|\\
Example: \verb|\Fext|\\
Output:
\begin{equation*}
    \Fext
\end{equation*}    

\noindent
Frame: \verb|\Frame{vector}{frame}|\\
Example: \verb|\Frame{A}{B}|\\
Output:
\begin{equation*}
    \Frame{A}{B}
\end{equation*}    

\noindent
Frame definition: \verb|\FrameDef{point}{frame}|\\
Example: \verb|\FrameDef{Q}{O}|\\
Output:
\begin{equation*}
    \FrameDef{Q}{O}
\end{equation*}  

\noindent
Frame w/ derivative: \verb|\FrameDeriv[<order>]{<frame>}{<var>}{<fun>}|\\
Inline version: \verb|\iFrameDeriv[<order>]{<frame>}{<var>}{<fun>}|\\
Example: \verb|\FrameDeriv[2]{A}{r}{t}|\\
Output:
\begin{equation*}
    \FrameDeriv[2]{A}{t}{r}
\end{equation*}  

\noindent
Inertia: \verb|\Inert{point}|\\
Example: \verb|\Inert{A}|\\
Output:
\begin{equation*}
    \Inert{A}
\end{equation*}  

\noindent
Inertia w/ frame: \verb|\InertF{point}{frame}|\\
Example: \verb|\InertF{A}{B}|\\
Output:
\begin{equation*}
    \InertF{A}{B}
\end{equation*}  

\noindent
Inertia tensor: \verb|\InertMat|\\
Example: \verb|\InertMat|\\
Output:
\begin{equation*}
    \InertMat
\end{equation*}  

\noindent
Lagrangian: \verb|\Lagr|\\
Example: \verb|\Lagr|\\
Output:
\begin{equation*}
    \Lagr
\end{equation*}  

\noindent
Omega matrix: \verb|\OmegaMat|\\
Example: \verb|\OmegaMat|\\
Output:
\begin{equation*}
    \OmegaMat
\end{equation*} 

\noindent
Partial derivative: \verb|\Partl[order]{with-respect-to-variable}|\\
Example: \verb|\Partl[]{x}|\\
Output:
\begin{equation*}
    \Partl[]{x}
\end{equation*} 
Example: \verb|\Partl[3]{t}|\\
Output:
\begin{equation*}
    \Partl[3]{t}
\end{equation*} 

\noindent
Partial framed derivative: \verb|\FramePartl[order]{frame}{with-respect-to-variable}|\\
Example: \verb|\FramePartl[2]{A}{x}|\\
Output:
\begin{equation*}
    \FramePartl[2]{A}{x}
\end{equation*} 

\noindent
Position vector: \verb|\PosVec{point}{with-respect-to-point}|\\
Example: \verb|\PosVec{A}{Q}|\\
Output:
\begin{equation*}
    \PosVec{A}{Q}
\end{equation*} 

\noindent
Quaternion angles: \verb|\QuatAngle|\\
Example: \verb|\QuatAngle|\\
Output:
\begin{equation*}
    \QuatAngle
\end{equation*} 

\noindent
Quaternion BCO matrix: \verb|\QuatBCO|\\
Example: \verb|\QuatBCO|\\
Output:
\begin{equation*}
    \QuatBCO
\end{equation*} 

\noindent
Quaternion definition: \verb|\QuatDef{frame}|\\
Example: \verb|\QuatDef{B}|\\
Output:
\begin{equation*}
    \QuatDef{B}
\end{equation*} 

\noindent
Quaternion derivative: \verb|\QuatDot|\\
Example: \verb|\QuatDot|\\
Output:
\begin{equation*}
    \QuatDot
\end{equation*} 

\noindent
Slanted fraction: \verb|\Rfrac{numerator}{denominator}|\\
Example: \verb|\Rfrac{A}{B}|\\
Output:
\begin{equation*}
    \Rfrac{A}{B}
\end{equation*} 

\noindent
Rotation matrix form: \verb|\Rotate{angle}{axis}|\\
Example: \verb|\Rotate{\phi}{X}|\\
Output:
\begin{equation*}
    \Rotate{\phi}{X}
\end{equation*} 

\noindent
Rotation matrix: \verb|\RotateMat{frame-1}{frame-2}|\\
Example: \verb|\RotateMat{B}{O}|\\
Output:
\begin{equation*}
    \RotateMat{B}{O}
\end{equation*} 

\noindent
Rotation matrix derivative: \verb|\RotateMatDot{frame-1}{frame-2}|\\
Example: \verb|\RotateMatDot{B}{O}|\\
Output:
\begin{equation*}
    \RotateMatDot{B}{O}
\end{equation*} 

\noindent
Rotation matrix about X: \verb|\RotateMatX{angle}|\\
Example: \verb|\RotateMatX{\theta}|\\
Output:
\begin{equation*}
    \RotateMatX{\theta}
\end{equation*} 

\noindent
Rotation matrix about Y: \verb|\RotateMatY{angle}|\\
Example: \verb|\RotateMatY{\theta}|\\
Output:
\begin{equation*}
    \RotateMatY{\theta}
\end{equation*} 

\noindent
Rotation matrix about Z: \verb|\RotateMatZ{angle}|\\
Example: \verb|\RotateMatZ{\theta}|\\
Output:
\begin{equation*}
    \RotateMatZ{\theta}
\end{equation*} 

\noindent
Sum of external torques: \verb|\Text[point]|\\
Example: \verb|\Text|\\
Output:
\begin{equation*}
    \Text
\end{equation*} 
Example: \verb|\Text[P]|\\
Output:
\begin{equation*}
    \Text[P]
\end{equation*} 

\noindent
1st transport theorem: \verb|\TransOne{frame-1}{frame-2}{vector}|\\
Inline version: \verb|\iTransOne{frame-1}{frame-2}{vector}|\\
Example: \verb|\TransOne{O}{B}{A}|\\
Output:
\begin{equation*}
    \TransOne{O}{B}{A}
\end{equation*} 

\noindent
2nd transport theorem: \verb|\TransTwo{frame-1}{frame-2}{vector}|\\
Inline version: \verb|\iTransTwo{frame-1}{frame-2}{vector}|\\
Example: \verb|\TransTwo{O}{B}{A}|\\
Output:
\begin{equation*}
    \TransTwo{O}{B}{A}
\end{equation*} 


\noindent
Dyad: \verb|\UnitDyad{unit-vector}{unit-vector}{frame}|\\
Example: \verb|\UnitDyad{i}{k}{B}|\\
Output:
\begin{equation*}
    \UnitDyad{i}{k}{B}
\end{equation*} 

\noindent
Unit vector: \verb|\UnitVec{vector}{frame}|\\
Example: \verb|\UnitVec{i}{O}|\\
Output:
\begin{equation*}
    \UnitVec{i}{O}
\end{equation*} 

\noindent
Left superscript frame: \verb|\UpRight{frame}{quantity}|\\
Example: \verb|\UpRight{O}{A}|\\
Output:
\begin{equation*}
    \UpRight{O}{A}
\end{equation*} 

\noindent
Horizontal vector: \verb|\VecExpressH{frame}{element-1}{element-2}{element-3}|\\
Example: \verb|\VecExpressH{D}{a}{b}{c}|\\
Output:
\begin{equation*}
    \VecExpressH{D}{a}{b}{c}
\end{equation*} 

\noindent
Vertical vector: \verb|\VecExpressV{element-1}{element-2}{element-3}|\\
Example: \verb|\VecExpressV{a}{b}{c}|\\
Output:
\begin{equation*}
    \VecExpressV{a}{b}{c}
\end{equation*} 

\noindent
4-element vertical vector: \verb|\VecExpressVF{element-1}{element-2}{element-3}{element-4}|\\
Example: \verb|\VecExpressVF{a}{b}{c}{d}|\\
Output:
\begin{equation*}
    \VecExpressVF{a}{b}{c}{d}
\end{equation*} 

\noindent
Velocity vector \verb|\VelVec{derivative-frame}{point}{with-respect-to-point}|\\
Example: \verb|\VelVec{O}{A}{Q}|\\
Output:
\begin{equation*}
    \VelVec{O}{A}{Q}
\end{equation*} 

\noindent
Angular velocity definition, X \verb|\wX{frame-1}{frame-2}|\\
Example: \verb|\wX{B}{O}|\\
Output:
\begin{equation*}
    \wX{B}{O}
\end{equation*} 

\noindent
Angular velocity definition, Y \verb|\wY{frame-1}{frame-2}|\\
Example: \verb|\wY{B}{O}|\\
Output:
\begin{equation*}
    \wY{B}{O}
\end{equation*} 

\noindent
Angular velocity definition, Z \verb|\wZ{frame-1}{frame-2}|\\
Example: \verb|\wZ{B}{O}|\\
Output:
\begin{equation*}
    \wZ{B}{O}
\end{equation*} 

% looking to fix the issues with the inline commands
\noindent The velocity vector $\VelVec{O}{A}{O}$ should be fine.\\
\noindent The transport theorem, rev2, is $\TransOne{O}{A}{\PosVec{B}{O}}$ \\
\noindent The transport theorem, rev3, is $\iTransOne{O}{A}{\PosVec{B}{O}}$\\

\noindent The transport theorem, rev2, is $\TransTwo{O}{A}{\PosVec{B}{O}}$ \\
\noindent The transport theorem, rev3, is $\iTransTwo{O}{A}{\PosVec{B}{O}}$


\end{document}
